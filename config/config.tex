\usepackage{algorithm}
\usepackage{algorithmicx}
\usepackage{algpseudocode}
\usepackage{amsfonts}
\usepackage{amsmath}
\usepackage{amsthm}
\usepackage{amssymb}
\usepackage{booktabs}
\usepackage{etoolbox}
\usepackage{float}
\usepackage{graphicx}
\usepackage{hyperref}
\usepackage{lipsum}
\usepackage{longtable}
\usepackage{minted}
\usepackage{pgffor}
\usepackage{subcaption}
\usepackage{tabularx}
\usepackage{tikz, pgfplots}
\usepackage{url}
\usepackage{xfrac}

\usepackage[english]{babel}
\usepackage[T1]{fontenc}
\usepackage[
    letterpaper,
    inner=1.25in,
    outer=1in,
    top=1in,
    bottom=1in
]{geometry}
\usepackage[scaled]{helvet}
\usepackage[utf8]{inputenc}
\usepackage[most]{tcolorbox}
% \usepackage[dvipsnames]{xcolor}

%%%%%%%%%%%%%%%%%%%%%%%%%%%%%%%%%%%%%%%%%%%%%%%%%%%%%%%%%%%%%%%%%%%%%%%%%%%%%%%%
% Colour
%%%%%%%%%%%%%%%%%%%%%%%%%%%%%%%%%%%%%%%%%%%%%%%%%%%%%%%%%%%%%%%%%%%%%%%%%%%%%%%%
% 2014 Material Design color palettes
% https://m2.material.io/design/color/the-color-system.html#tools-for-picking-colors

\definecolor{Red-50}          {HTML}{FFEBEE}
\definecolor{Red-100}         {HTML}{FFCDD2}
\definecolor{Red-200}         {HTML}{EF9A9A}
\definecolor{Red-300}         {HTML}{E57373}
\definecolor{Red-400}         {HTML}{EF5350}
\definecolor{Red-500}         {HTML}{F44336}
\definecolor{Red-600}         {HTML}{E53935}
\definecolor{Red-700}         {HTML}{D32F2F}
\definecolor{Red-800}         {HTML}{C62828}
\definecolor{Red-900}         {HTML}{B71C1C}
\definecolor{Red-A100}        {HTML}{FF8A80}
\definecolor{Red-A200}        {HTML}{FF5252}
\definecolor{Red-A400}        {HTML}{FF1744}
\definecolor{Red-A700}        {HTML}{D50000}

\definecolor{Pink-50}         {HTML}{FCE4EC}
\definecolor{Pink-100}        {HTML}{F8BBD0}
\definecolor{Pink-200}        {HTML}{F48FB1}
\definecolor{Pink-300}        {HTML}{F06292}
\definecolor{Pink-400}        {HTML}{EC407A}
\definecolor{Pink-500}        {HTML}{E91E63}
\definecolor{Pink-600}        {HTML}{D81B60}
\definecolor{Pink-700}        {HTML}{C2185B}
\definecolor{Pink-800}        {HTML}{AD1457}
\definecolor{Pink-900}        {HTML}{880E4F}
\definecolor{Pink-A100}       {HTML}{FF80AB}
\definecolor{Pink-A200}       {HTML}{FF4081}
\definecolor{Pink-A400}       {HTML}{F50057}
\definecolor{Pink-A700}       {HTML}{C51162}

\definecolor{Purple-50}       {HTML}{F3E5F5}
\definecolor{Purple-100}      {HTML}{E1BEE7}
\definecolor{Purple-200}      {HTML}{CE93D8}
\definecolor{Purple-300}      {HTML}{BA68C8}
\definecolor{Purple-400}      {HTML}{AB47BC}
\definecolor{Purple-500}      {HTML}{9C27B0}
\definecolor{Purple-600}      {HTML}{8E24AA}
\definecolor{Purple-700}      {HTML}{7B1FA2}
\definecolor{Purple-800}      {HTML}{6A1B9A}
\definecolor{Purple-900}      {HTML}{4A148C}
\definecolor{Purple-A100}     {HTML}{EA80FC}
\definecolor{Purple-A200}     {HTML}{E040FB}
\definecolor{Purple-A400}     {HTML}{D500F9}
\definecolor{Purple-A700}     {HTML}{AA00FF}

\definecolor{Deep-Purple-50}  {HTML}{EDE7F6}
\definecolor{Deap-Purple-100} {HTML}{D1C4E9}
\definecolor{Deap-Purple-200} {HTML}{B39DDB}
\definecolor{Deap-Purple-300} {HTML}{9575CD}
\definecolor{Deap-Purple-400} {HTML}{7E57C2}
\definecolor{Deap-Purple-500} {HTML}{673AB7}
\definecolor{Deap-Purple-600} {HTML}{5E35B1}
\definecolor{Deap-Purple-700} {HTML}{512DA8}
\definecolor{Deap-Purple-800} {HTML}{4527A0}
\definecolor{Deap-Purple-900} {HTML}{311B92}
\definecolor{Deap-Purple-A100}{HTML}{B388FF}
\definecolor{Deap-Purple-A200}{HTML}{7C4DFF}
\definecolor{Deap-Purple-A400}{HTML}{651FFF}
\definecolor{Deap-Purple-A700}{HTML}{6200EA}

\definecolor{Indigo-50}       {HTML}{E8EAF6}
\definecolor{Indigo-100}      {HTML}{C5CAE9}
\definecolor{Indigo-200}      {HTML}{9FA8DA}
\definecolor{Indigo-300}      {HTML}{7986CB}
\definecolor{Indigo-400}      {HTML}{5C6BC0}
\definecolor{Indigo-500}      {HTML}{3F51B5}
\definecolor{Indigo-600}      {HTML}{3949AB}
\definecolor{Indigo-700}      {HTML}{303F9F}
\definecolor{Indigo-800}      {HTML}{283593}
\definecolor{Indigo-900}      {HTML}{1A237E}
\definecolor{Indigo-A100}     {HTML}{8C9EFF}
\definecolor{Indigo-A200}     {HTML}{536DFE}
\definecolor{Indigo-A400}     {HTML}{3D5AFE}
\definecolor{Indigo-A700}     {HTML}{304FFE}

\definecolor{Blue-50}         {HTML}{E3F2FD}
\definecolor{Blue-100}        {HTML}{BBDEFB}
\definecolor{Blue-200}        {HTML}{90CAF9}
\definecolor{Blue-300}        {HTML}{64B5F6}
\definecolor{Blue-400}        {HTML}{42A5F5}
\definecolor{Blue-500}        {HTML}{2196F3}
\definecolor{Blue-600}        {HTML}{1E88E5}
\definecolor{Blue-700}        {HTML}{1976D2}
\definecolor{Blue-800}        {HTML}{1565C0}
\definecolor{Blue-900}        {HTML}{0D47A1}
\definecolor{Blue-A100}       {HTML}{82B1FF}
\definecolor{Blue-A200}       {HTML}{448AFF}
\definecolor{Blue-A400}       {HTML}{2979FF}
\definecolor{Blue-A700}       {HTML}{2962FF}

\definecolor{Light-Blue-50}   {HTML}{E1F5FE}
\definecolor{Light-Blue-100}  {HTML}{B3E5FC}
\definecolor{Light-Blue-200}  {HTML}{81D4FA}
\definecolor{Light-Blue-300}  {HTML}{4FC3F7}
\definecolor{Light-Blue-400}  {HTML}{29B6F6}
\definecolor{Light-Blue-500}  {HTML}{03A9F4}
\definecolor{Light-Blue-600}  {HTML}{039BE5}
\definecolor{Light-Blue-700}  {HTML}{0288D1}
\definecolor{Light-Blue-800}  {HTML}{0277BD}
\definecolor{Light-Blue-900}  {HTML}{01579B}
\definecolor{Light-Blue-A100} {HTML}{80D8FF}
\definecolor{Light-Blue-A200} {HTML}{40C4FF}
\definecolor{Light-Blue-A400} {HTML}{00B0FF}
\definecolor{Light-Blue-A700} {HTML}{0091EA}

\definecolor{Cyan-50}         {HTML}{E0F7FA}
\definecolor{Cyan-100}        {HTML}{B2EBF2}
\definecolor{Cyan-200}        {HTML}{80DEEA}
\definecolor{Cyan-300}        {HTML}{4DD0E1}
\definecolor{Cyan-400}        {HTML}{26C6DA}
\definecolor{Cyan-500}        {HTML}{00BCD4}
\definecolor{Cyan-600}        {HTML}{00ACC1}
\definecolor{Cyan-700}        {HTML}{0097A7}
\definecolor{Cyan-800}        {HTML}{00838F}
\definecolor{Cyan-900}        {HTML}{006064}
\definecolor{Cyan-A100}       {HTML}{84FFFF}
\definecolor{Cyan-A200}       {HTML}{18FFFF}
\definecolor{Cyan-A400}       {HTML}{00E5FF}
\definecolor{Cyan-A700}       {HTML}{00B8D4}

\definecolor{Teal-50}         {HTML}{E0F2F1}
\definecolor{Teal-100}        {HTML}{B2DFDB}
\definecolor{Teal-200}        {HTML}{80CBC4}
\definecolor{Teal-300}        {HTML}{4DB6AC}
\definecolor{Teal-400}        {HTML}{26A69A}
\definecolor{Teal-500}        {HTML}{009688}
\definecolor{Teal-600}        {HTML}{00897B}
\definecolor{Teal-700}        {HTML}{00796B}
\definecolor{Teal-800}        {HTML}{00695C}
\definecolor{Teal-900}        {HTML}{004D40}
\definecolor{Teal-A100}       {HTML}{A7FFEB}
\definecolor{Teal-A200}       {HTML}{64FFDA}
\definecolor{Teal-A400}       {HTML}{1DE9B6}
\definecolor{Teal-A700}       {HTML}{00BFA5}

\definecolor{Green-50}        {HTML}{E8F5E9}
\definecolor{Green-100}       {HTML}{C8E6C9}
\definecolor{Green-200}       {HTML}{A5D6A7}
\definecolor{Green-300}       {HTML}{81C784}
\definecolor{Green-400}       {HTML}{66BB6A}
\definecolor{Green-500}       {HTML}{4CAF50}
\definecolor{Green-600}       {HTML}{43A047}
\definecolor{Green-700}       {HTML}{388E3C}
\definecolor{Green-800}       {HTML}{2E7D32}
\definecolor{Green-900}       {HTML}{1B5E20}
\definecolor{Green-A100}      {HTML}{B9F6CA}
\definecolor{Green-A200}      {HTML}{69F0AE}
\definecolor{Green-A400}      {HTML}{00E676}
\definecolor{Green-A700}      {HTML}{00C853}

\definecolor{Light-Green-50}  {HTML}{F1F8E9}
\definecolor{Light-Green-100} {HTML}{DCEDC8}
\definecolor{Light-Green-200} {HTML}{C5E1A5}
\definecolor{Light-Green-300} {HTML}{AED581}
\definecolor{Light-Green-400} {HTML}{9CCC65}
\definecolor{Light-Green-500} {HTML}{8BC34A}
\definecolor{Light-Green-600} {HTML}{7CB342}
\definecolor{Light-Green-700} {HTML}{689F38}
\definecolor{Light-Green-800} {HTML}{558B2F}
\definecolor{Light-Green-900} {HTML}{33691E}
\definecolor{Light-Green-A100}{HTML}{CCFF90}
\definecolor{Light-Green-A200}{HTML}{B2FF59}
\definecolor{Light-Green-A400}{HTML}{76FF03}
\definecolor{Light-Green-A700}{HTML}{64DD17}

\definecolor{Lime-50}         {HTML}{F9FBE7}
\definecolor{Lime-100}        {HTML}{F0F4C3}
\definecolor{Lime-200}        {HTML}{E6EE9C}
\definecolor{Lime-300}        {HTML}{DCE775}
\definecolor{Lime-400}        {HTML}{D4E157}
\definecolor{Lime-500}        {HTML}{CDDC39}
\definecolor{Lime-600}        {HTML}{C0CA33}
\definecolor{Lime-700}        {HTML}{AFB42B}
\definecolor{Lime-800}        {HTML}{9E9D24}
\definecolor{Lime-900}        {HTML}{827717}
\definecolor{Lime-A100}       {HTML}{F4FF81}
\definecolor{Lime-A200}       {HTML}{EEFF41}
\definecolor{Lime-A400}       {HTML}{C6FF00}
\definecolor{Lime-A700}       {HTML}{AEEA00}

\definecolor{Yellow-50}       {HTML}{FFFDE7}
\definecolor{Yellow-100}      {HTML}{FFF9C4}
\definecolor{Yellow-200}      {HTML}{FFF59D}
\definecolor{Yellow-300}      {HTML}{FFF176}
\definecolor{Yellow-400}      {HTML}{FFEE58}
\definecolor{Yellow-500}      {HTML}{FFEB3B}
\definecolor{Yellow-600}      {HTML}{FDD835}
\definecolor{Yellow-700}      {HTML}{FBC02D}
\definecolor{Yellow-800}      {HTML}{F9A825}
\definecolor{Yellow-900}      {HTML}{F57F17}
\definecolor{Yellow-A100}     {HTML}{FFFF8D}
\definecolor{Yellow-A200}     {HTML}{FFFF00}
\definecolor{Yellow-A400}     {HTML}{FFEA00}
\definecolor{Yellow-A700}     {HTML}{FFD600}

\definecolor{Amber-50}        {HTML}{FFF8E1}
\definecolor{Amber-100}       {HTML}{FFECB3}
\definecolor{Amber-200}       {HTML}{FFE082}
\definecolor{Amber-300}       {HTML}{FFD54F}
\definecolor{Amber-400}       {HTML}{FFCA28}
\definecolor{Amber-500}       {HTML}{FFC107}
\definecolor{Amber-600}       {HTML}{FFB300}
\definecolor{Amber-700}       {HTML}{FFA000}
\definecolor{Amber-800}       {HTML}{FF8F00}
\definecolor{Amber-900}       {HTML}{FF6F00}
\definecolor{Amber-A100}      {HTML}{FFE57F}
\definecolor{Amber-A200}      {HTML}{FFD740}
\definecolor{Amber-A400}      {HTML}{FFC400}
\definecolor{Amber-A700}      {HTML}{FFAB00}

\definecolor{Orange-50}       {HTML}{FFF3E0}
\definecolor{Orange-100}      {HTML}{FFE0B2}
\definecolor{Orange-200}      {HTML}{FFCC80}
\definecolor{Orange-300}      {HTML}{FFB74D}
\definecolor{Orange-400}      {HTML}{FFA726}
\definecolor{Orange-500}      {HTML}{FF9800}
\definecolor{Orange-600}      {HTML}{FB8C00}
\definecolor{Orange-700}      {HTML}{F57C00}
\definecolor{Orange-800}      {HTML}{EF6C00}
\definecolor{Orange-900}      {HTML}{E65100}
\definecolor{Orange-A100}     {HTML}{FFD180}
\definecolor{Orange-A200}     {HTML}{FFAB40}
\definecolor{Orange-A400}     {HTML}{FF9100}
\definecolor{Orange-A700}     {HTML}{FF6D00}

\definecolor{Deep-Orange-50  }{HTML}{FBE9E7}
\definecolor{Deep-Orange-100 }{HTML}{FFCCBC}
\definecolor{Deep-Orange-200 }{HTML}{FFAB91}
\definecolor{Deep-Orange-300 }{HTML}{FF8A65}
\definecolor{Deep-Orange-400 }{HTML}{FF7043}
\definecolor{Deep-Orange-500 }{HTML}{FF5722}
\definecolor{Deep-Orange-600 }{HTML}{F4511E}
\definecolor{Deep-Orange-700 }{HTML}{E64A19}
\definecolor{Deep-Orange-800 }{HTML}{D84315}
\definecolor{Deep-Orange-900 }{HTML}{BF360C}
\definecolor{Deep-Orange-A100}{HTML}{FF9E80}
\definecolor{Deep-Orange-A200}{HTML}{FF6E40}
\definecolor{Deep-Orange-A400}{HTML}{FF3D00}
\definecolor{Deep-Orange-A700}{HTML}{DD2C00}

\definecolor{Brown-50}        {HTML}{EFEBE9}
\definecolor{Brown-100}       {HTML}{D7CCC8}
\definecolor{Brown-200}       {HTML}{BCAAA4}
\definecolor{Brown-300}       {HTML}{A1887F}
\definecolor{Brown-400}       {HTML}{8D6E63}
\definecolor{Brown-500}       {HTML}{795548}
\definecolor{Brown-600}       {HTML}{6D4C41}
\definecolor{Brown-700}       {HTML}{5D4037}
\definecolor{Brown-800}       {HTML}{4E342E}
\definecolor{Brown-900}       {HTML}{3E2723}

\definecolor{Gray-50}         {HTML}{FAFAFA}
\definecolor{Gray-100}        {HTML}{F5F5F5}
\definecolor{Gray-200}        {HTML}{EEEEEE}
\definecolor{Gray-300}        {HTML}{E0E0E0}
\definecolor{Gray-400}        {HTML}{BDBDBD}
\definecolor{Gray-500}        {HTML}{9E9E9E}
\definecolor{Gray-600}        {HTML}{757575}
\definecolor{Gray-700}        {HTML}{616161}
\definecolor{Gray-800}        {HTML}{424242}
\definecolor{Gray-900}        {HTML}{212121}

\definecolor{Blue-Gray-50}    {HTML}{ECEFF1}
\definecolor{Blue-Gray-100}   {HTML}{CFD8DC}
\definecolor{Blue-Gray-200}   {HTML}{B0BEC5}
\definecolor{Blue-Gray-300}   {HTML}{90A4AE}
\definecolor{Blue-Gray-400}   {HTML}{78909C}
\definecolor{Blue-Gray-500}   {HTML}{607D8B}
\definecolor{Blue-Gray-600}   {HTML}{546E7A}
\definecolor{Blue-Gray-700}   {HTML}{455A64}
\definecolor{Blue-Gray-800}   {HTML}{37474F}
\definecolor{Blue-Gray-900}   {HTML}{263238}

\definecolor{Black}           {HTML}{000000}
\definecolor{White}           {HTML}{FFFFFF}


% Light-Blue-700
\definecolor{primary}{HTML}{0288D1}

% Hyperref ---------------------------------------------------------------------
\hypersetup{
    colorlinks,
    linkcolor=black, 
    filecolor=Pink-300,
    urlcolor=Blue-700,
    citecolor=black,
}

%%%%%%%%%%%%%%%%%%%%%%%%%%%%%%%%%%%%%%%%%%%%%%%%%%%%%%%%%%%%%%%%%%%%%%%%%%%%%%%%
% Custom commands
%%%%%%%%%%%%%%%%%%%%%%%%%%%%%%%%%%%%%%%%%%%%%%%%%%%%%%%%%%%%%%%%%%%%%%%%%%%%%%%%
\newcommand{\by}[1]{\hfill--- #1}

\makeatletter
\renewcommand*\env@matrix[1][*\c@MaxMatrixCols c]{%
    \hskip -\arraycolsep
    \let\@ifnextchar\new@ifnextchar
    \array{#1}}
\makeatother

% TODO: Add more custom commands here
\newcommand{\R}{\mathbb{R}}
\renewcommand{\P}{\mathbb{P}}

\renewcommand{\vec}[1]{\underline{#1}}

\pgfplotsset{compat=1.18}

%%%%%%%%%%%%%%%%%%%%%%%%%%%%%%%%%%%%%%%%%%%%%%%%%%%%%%%%%%%%%%%%%%%%%%%%%%%%%%%%
% Text styles
%%%%%%%%%%%%%%%%%%%%%%%%%%%%%%%%%%%%%%%%%%%%%%%%%%%%%%%%%%%%%%%%%%%%%%%%%%%%%%%%
% Font -------------------------------------------------------------------------
% \renewcommand{\familydefault}{\sfdefault} % Sans-serif font
\renewcommand{\familydefault}{\rmdefault} % Serif font

% Text styles ------------------------------------------------------------------
\DeclareTextFontCommand{\term}{\color{Orange-700}\bfseries}

\DeclareTextFontCommand{\bold}{\bfseries}
\DeclareTextFontCommand{\italic}{\itshape}

\renewcommand{\textbf}[1]{\textcolor{Red-800}{\bfseries #1}}
\renewcommand{\textit}[1]{\textcolor{Light-Blue-400}{\itshape #1}}

% Paragraph spacing ------------------------------------------------------------
% \setlength{\parindent}{0pt} % No paragraph indentation
% \setlength{\parskip}{1em} % Paragraph spacing

%%%%%%%%%%%%%%%%%%%%%%%%%%%%%%%%%%%%%%%%%%%%%%%%%%%%%%%%%%%%%%%%%%%%%%%%%%%%%%%%
% Environments
%%%%%%%%%%%%%%%%%%%%%%%%%%%%%%%%%%%%%%%%%%%%%%%%%%%%%%%%%%%%%%%%%%%%%%%%%%%%%%%%
\renewenvironment{quote}
{\list{}{\rightmargin=0.5cm\leftmargin=0.5cm}\item\relax\itshape}
{\endlist}

% Lists ------------------------------------------------------------------------
% itemize
\renewcommand{\labelitemi}{\textcolor{primary}{\textbullet}}
\renewcommand{\labelitemii}{\textcolor{primary}{\textbullet}}
\renewcommand{\labelitemiii}{\textcolor{primary}{\textbullet}}
\renewcommand{\labelitemiv}{\textcolor{primary}{\textbullet}}

% enumerate
\newcommand{\cnum}[2]{
    \tikz[baseline=(c.base)]
    \node[circle,fill=#2,minimum size=0.5cm,inner sep=1pt](c)
    {\color{white}\bfseries\fontsize{8}{8}#1};}

\renewcommand{\labelenumi}{\cnum{\arabic{enumi}}{primary}}
\renewcommand{\labelenumii}{\cnum{\alph{enumii}}{primary}}
\renewcommand{\labelenumiii}{\cnum{\roman{enumiii}}{primary}}
\renewcommand{\labelenumiv}{\cnum{\Alph{enumiv}}{primary}}

% amsthm -----------------------------------------------------------------------
\newcommand\fancybox[3]{%
    \tcbset{mybox/.style={enhanced,boxsep=0mm,opacityfill=0,overlay={
                    \coordinate (X) at ([xshift=-1mm, yshift=-1.5mm]frame.north west);
                    \node[align=right, text=#1, text width=2.5cm, anchor=north east] at (X) {\bf#2};
                    \draw[line width=0.5mm, color=#1] (frame.north west) -- (frame.south west);
                }}} \begin{tcolorbox}[mybox] #3 \end{tcolorbox}}

\tcbuselibrary{theorems,skins,hooks}
\NewDocumentCommand\thmbox{m O{\Large #1} O{Gray-100} O{primary} O{number within=section}}
{
    \newtcbtheorem[#5]{#1}{\large #2}
    {%
        enhanced,
        breakable,
        colback = #3,
        frame hidden,
        boxrule = 0sp,
        borderline west = {2pt}{0pt}{#4},
        sharp corners,
        detach title,
        before upper = \tcbtitle\par\smallskip,
        coltitle = #4,
        fonttitle = \bfseries,
        description font = \mdseries,
        separator sign none,
        segmentation style={solid, #4}
    }
    {th}
}

% Universal counter for theorem environments
\newcounter{universal}

% Corollary --------------------------------------------------------------------
\thmbox{Corollary} [Corollary] [Purple-50] [Purple-800] 
[use counter=universal, number within=section]

\newenvironment{corollary}[1][] {\begin{Corollary}{#1}{}} {\end{Corollary}}

% Definition -------------------------------------------------------------------
\thmbox{Definition} [Definition] [Light-Blue-50] [Light-Blue-800]
[use counter=universal, number within=section]

\newenvironment{definition}[1][] {\begin{Definition}{#1}{}} {\end{Definition}}

% Example ----------------------------------------------------------------------
\theoremstyle{definition}
\newtheorem*{Example}{\color{primary}Example}

\newenvironment{example}
{\begin{Example}}
% {\begin{Example}\setlength{\parindent}{0pt}}
{\hfill\ensuremath{\color{primary}\diamondsuit}\end{Example}}

% Lemma ------------------------------------------------------------------------
\thmbox{Lemma} [Lemma] [Green-50] [Green-800]
[use counter=universal, number within=section]

\newenvironment{lemma}[1][] {\begin{Lemma}{#1}{}} {\end{Lemma}}

% Remark -----------------------------------------------------------------------
\thmbox{Remark} [Remark] [Gray-200] [Gray-700] [no counter]

\newenvironment{remark}[1][] {\begin{Remark}{#1}{}} {\end{Remark}}

% Proposition ------------------------------------------------------------------
\thmbox{Proposition} [Proposition] [Orange-50] [Orange-800]
[use counter=universal, number within=section]

\newenvironment{proposition}[1][] {\begin{Proposition}{#1}{}} {\end{Proposition}}

% Theorem ----------------------------------------------------------------------
\thmbox{Theorem} [Theorem] [Red-50] [Red-800]
[use counter=universal, number within=section]

\newenvironment{theorem}[1][] {\begin{Theorem}{#1}{}} {\end{Theorem}}

% Note -------------------------------------------------------------------------
\newtcolorbox{note}[1][] {
    colback=Gray-100,
    colframe=Gray-200,
    coltitle=Gray-700,
    title=\bold{Note: #1},
    fonttitle=\bfseries,
    breakable
}

% Algorithms -------------------------------------------------------------------
% \Call
\algrenewcommand\Call[2]{\textproc{\color{primary}\textsc{#1}}(#2)}

% \Else
\algrenewcommand\algorithmicelse {\textsc{\color{primary}Else}}

% \For
\algrenewcommand\algorithmicfor {\textsc{\color{primary}For}}
\algtext*{EndFor}

% \Function
\algrenewcommand\algorithmicfunction {\textsc{\color{primary}Function}}
\algtext*{EndFunction}

% \If
\algrenewcommand\algorithmicif {\textsc{\color{primary}If}}
\algtext*{EndIf}

% \Return
\algrenewcommand\algorithmicreturn {\textsc{\color{primary}Return}}

% \To
\newcommand{\To}{\textsc{\color{primary}To}\xspace}

% \DownTo
\newcommand{\DownTo}{\textsc{\color{primary}DownTo}\xspace}

% \While
\algrenewcommand\algorithmicwhile {\textsc{\color{primary}While}}
\algtext*{EndWhile}

% Proof ------------------------------------------------------------------------
\renewcommand{\qedsymbol}{\ensuremath\blacksquare}

%%%%%%%%%%%%%%%%%%%%%%%%%%%%%%%%%%%%%%%%%%%%%%%%%%%%%%%%%%%%%%%%%%%%%%%%%%%%%%%%
% Heading styles
%%%%%%%%%%%%%%%%%%%%%%%%%%%%%%%%%%%%%%%%%%%%%%%%%%%%%%%%%%%%%%%%%%%%%%%%%%%%%%%%
\usepackage[explicit]{titlesec}

% Part -------------------------------------------------------------------------
\titleformat{\part}[display]
{\Large\bf} {\textsc{\partname~\thepart}} {12pt} {
    \Huge\textsc{#1}
} [\thispagestyle{empty}]

% Chapter ----------------------------------------------------------------------
\newtcolorbox{titlecolorbox}[1]{
    coltext=white,
    colframe=primary,
    colback=primary,
    boxrule=0pt,
    arc=0pt,
    notitle,
    width=4.3em,
    height=2.4ex,
    before=\hfill
}

\makeatletter
\let\old@rule\@rule
\def\@rule[#1]#2#3{\textcolor{primary}{\old@rule[#1]{#2}{#3}}}
\makeatother

\titleformat{\chapter}[display]
{\Huge} {} {0pt} {
    \begin{titlecolorbox}{}
        {\large\MakeUppercase{\bf\chaptername}}
    \end{titlecolorbox}
    \vspace*{-3.19ex}\noindent\rule{\textwidth}{0.4pt}
    \parbox[b]{\dimexpr\textwidth-4.8em\relax}{\raggedright\textsc{#1}}
    {\hfill\fontsize{50}{40}\selectfont{\color{primary}\thechapter}}
} [\thispagestyle{empty}]

\titleformat{name=\chapter,numberless}[display]
{\Huge} {} {0pt} {
    \vspace*{-3.19ex}\noindent\rule{\textwidth}{0.4pt}
    \parbox[b]{\dimexpr\textwidth-4.8em\relax}{\raggedright\MakeUppercase{#1}}
} [\thispagestyle{empty}]

\titlespacing*{\chapter}{0pt}{-20pt}{20pt}

% Section ----------------------------------------------------------------------
\titleformat{\section}[hang]{\Large\bfseries}%
{\rlap{ \color{primary} \rule[-6pt] {\textwidth} {0.4pt} }
    \colorbox{primary} {
        \raisebox{0pt}[13pt][3pt] {
            \makebox[60pt]{ \selectfont\color{white}{\thesection} }
        }
    }
} {15pt} {\color{primary}#1}

% Subsection -------------------------------------------------------------------
\titleformat{\subsection}[hang]{\Large\bfseries}
{\color{primary}{\thesubsection}} {10pt} {\color{primary}#1}

%%%%%%%%%%%%%%%%%%%%%%%%%%%%%%%%%%%%%%%%%%%%%%%%%%%%%%%%%%%%%%%%%%%%%%%%%%%%%%%%
% Header and footer
%%%%%%%%%%%%%%%%%%%%%%%%%%%%%%%%%%%%%%%%%%%%%%%%%%%%%%%%%%%%%%%%%%%%%%%%%%%%%%%%
\usepackage{fancyhdr}

\fancypagestyle{fancy} {
    % Clear header/footer
    \fancyhf{}
    % Page number (left of even, right of odd)
    \fancyhead[LE,RO]{\thepage}
    % Chapter name (right of even)
    \fancyhead[RE]{\nouppercase{\leftmark}}
    % Section name (left of odd)
    \fancyhead[LO]{\nouppercase{\rightmark}}
    % No header rule
    \renewcommand{\headrulewidth}{0pt}
}


% Define new page style for frontmatter
\fancypagestyle{frontmatter} {
    % Clear header/footer
    \fancyhf{}
    % No header rule
    \renewcommand{\headrulewidth}{0pt}
    % Page in footer, centred
    \fancyfoot[C]{\thepage}
}

%%%%%%%%%%%%%%%%%%%%%%%%%%%%%%%%%%%%%%%%%%%%%%%%%%%%%%%%%%%%%%%%%%%%%%%%%%%%%%%%
% Table of contents
%%%%%%%%%%%%%%%%%%%%%%%%%%%%%%%%%%%%%%%%%%%%%%%%%%%%%%%%%%%%%%%%%%%%%%%%%%%%%%%%
\usepackage{blindtext}
\usepackage{framed}
\usepackage{titletoc}

\patchcmd{\tableofcontents}{\contentsname}{\contentsname}{}{}

\newtoggle{isUnnumberedChapter}
\togglefalse{isUnnumberedChapter} % Default state

\renewenvironment{leftbar}
{\def\FrameCommand{\hspace{6em}%
        {\color{primary}\vrule width 2pt depth 6pt}\hspace{1em}}%
    \MakeFramed{\parshape 1 0cm \dimexpr\textwidth-6em\relax\FrameRestore}\vskip2pt%
}
{\endMakeFramed}

\titlecontents{chapter}[0em]
{\vspace*{2\baselineskip}}
{\parbox{4.5em}{%
        \hfill\Huge\bfseries\color{primary}\thecontentslabel}%
    \vspace*{-2.3\baselineskip}\leftbar\textbf{\color{primary}\small\chaptername~\thecontentslabel}\\
}{}{\endleftbar}

\titlecontents{section}[8.4em]
{\contentslabel{3em}}{}{}
{\hspace{0.5em}\nobreak\itshape\color{primary}\contentspage}

\titlecontents{subsection}[11.4em]
{\contentslabel{3em}}{}{}
{\hspace{0.5em}\nobreak\itshape\color{primary}\contentspage}

%%%%%%%%%%%%%%%%%%%%%%%%%%%%%%%%%%%%%%%%%%%%%%%%%%%%%%%%%%%%%%%%%%%%%%%%%%%%%%%%
% Glossary
%%%%%%%%%%%%%%%%%%%%%%%%%%%%%%%%%%%%%%%%%%%%%%%%%%%%%%%%%%%%%%%%%%%%%%%%%%%%%%%%
% TODO: Glossary configuration

%%%%%%%%%%%%%%%%%%%%%%%%%%%%%%%%%%%%%%%%%%%%%%%%%%%%%%%%%%%%%%%%%%%%%%%%%%%%%%%%
% Bibliography
%%%%%%%%%%%%%%%%%%%%%%%%%%%%%%%%%%%%%%%%%%%%%%%%%%%%%%%%%%%%%%%%%%%%%%%%%%%%%%%%
\usepackage{csquotes}

\usepackage[
    style=ieee,
    citestyle=ieee,
    sorting=nyt,
    sortcites=true,
    autopunct=true,
    autolang=hyphen,
    hyperref=true,
    abbreviate=false,
    backref=true,
    backend=biber,
    defernumbers=true
]{biblatex}

\addbibresource{bibliography/template.bib}
\addbibresource{bibliography/reference.bib}
\defbibheading{bibempty}{}

%%%%%%%%%%%%%%%%%%%%%%%%%%%%%%%%%%%%%%%%%%%%%%%%%%%%%%%%%%%%%%%%%%%%%%%%%%%%%%%%
% Index
%%%%%%%%%%%%%%%%%%%%%%%%%%%%%%%%%%%%%%%%%%%%%%%%%%%%%%%%%%%%%%%%%%%%%%%%%%%%%%%%
\usepackage{calc}
\usepackage{makeidx}

\makeindex

%%%%%%%%%%%%%%%%%%%%%%%%%%%%%%%%%%%%%%%%%%%%%%%%%%%%%%%%%%%%%%%%%%%%%%%%%%%%%%%%
% Tikz Externalize
%%%%%%%%%%%%%%%%%%%%%%%%%%%%%%%%%%%%%%%%%%%%%%%%%%%%%%%%%%%%%%%%%%%%%%%%%%%%%%%%
\usetikzlibrary{external}
\tikzexternalize[prefix=tikz/]

% Disable externalization globally, and only enable it for `tikzpicture'
\tikzexternaldisable
\BeforeBeginEnvironment[label]{tikzpicture}{\tikzexternalenable}
\AfterEndEnvironment[label]{tikzpicture}{\tikzexternaldisable}

%%%%%%%%%%%%%%%%%%%%%%%%%%%%%%%%%%%%%%%%%%%%%%%%%%%%%%%%%%%%%%%%%%%%%%%%%%%%%%%%
% Title page
%%%%%%%%%%%%%%%%%%%%%%%%%%%%%%%%%%%%%%%%%%%%%%%%%%%%%%%%%%%%%%%%%%%%%%%%%%%%%%%%
\usetikzlibrary{calc}
\usetikzlibrary{shapes.geometric}
\usepackage{anyfontsize}
\newcommand{\frontpage}[3]{
    \tikzset{external/export next=false}
    \begin{tikzpicture}[remember picture, overlay]
        % Background
        \fill[primary] (current page.south west) rectangle (current page.north east);

        \foreach \i in {2.5,...,22} {
            \node[rounded corners,primary!60,draw,regular polygon,regular polygon sides=6, minimum size=\i cm,ultra thick] at ($(current page.west)+(2.5,-5)$) {} ;
        }

        % Background Polygon
        \foreach \i in {0.5,...,22} {
            \node[rounded corners,primary!60,draw,regular polygon,regular polygon sides=6, minimum size=\i cm,ultra thick] at ($(current page.north west)+(2.5,0)$) {} ;
        }

        \foreach \i in {0.5,...,22} {
            \node[rounded corners,primary!90,draw,regular polygon,regular polygon sides=6, minimum size=\i cm,ultra thick] at ($(current page.north east)+(0,-9.5)$) {} ;
        }

        \foreach \i in {21,...,6} {
            \node[primary!85,rounded corners,draw,regular polygon,regular polygon sides=6, minimum size=\i cm,ultra thick] at ($(current page.south east)+(-0.2,-0.45)$) {} ;
        }

        % Title
        \node[left,primary!5,minimum width=0.625*\paperwidth,minimum height=3cm, rounded corners] at ($(current page.north east)+(0,-9.5)$) {
            {\fontsize{25}{30} \selectfont \bfseries #1}
        };

        % Subtitle
        \node[left,primary!10,minimum width=0.625*\paperwidth,minimum height=2cm, rounded corners] at ($(current page.north east)+(0,-11)$) {
            {\huge \italic{#2}}
        };

        % Author
        \node[left,primary!5,minimum width=0.625*\paperwidth,minimum height=2cm, rounded corners] at ($(current page.north east)+(0,-13)$) {
            {\Large \textsc{#3}}
        };

        % Year
        \node[rounded corners,fill=primary!70,text =primary!5,regular polygon,regular polygon sides=6, minimum size=2.5 cm,inner sep=0,ultra thick] at ($(current page.west)+(2.5,-5)$) {\LARGE \bfseries \the\year{}};
    \end{tikzpicture}
}