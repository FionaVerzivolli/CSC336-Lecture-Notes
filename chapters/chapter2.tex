\chapter{Computer Arithmetic and Computational Errors}

\section{Numerical Stability}

There is only finite space in computer. How would we store \( \pi \), an irrational number? We can't. We can only store an approximation of \( \pi \). How does introduction of approximations affect the accuracy of our computations?

\begin{example}
    Suppose we want to compute the value for the sequence of integrals \[
        y_n = \int_0^1 \frac{x^n}{x + 5} \, dx
    \] for \( n = 0, 1, 2, \ldots, 8 \), with 3 decimal digits of accuracy.

    There are several properties that I can claim:

    \begin{itemize}
        \item \( y_n > 0 \) for all \( n \), since the integrand \( \frac{x^n}{x + 5} > 0 \) for all \( x \in (0, 1) \).
        \item \( y_{n+1} < y_n \) for all \( n \), since the integrand \( \frac{x^{n+1}}{x + 5}
              = x \cdot \frac{x^n}{x + 5} < \frac{x^n}{x + 5} \) for all \( x \in (0, 1) \).
    \end{itemize}

    There is not closed-form solution to this problem.

    \begin{align*}
        x^n
         & = x^n \cdot \frac{x + 5}{x + 5}
         & \text{for } x \in (0, 1)                                                  \\
        x^n
         & = \frac{x^{n+1}}{x + 5} + \frac{5x^n}{x + 5}                              \\
        \int_0^1 x^n \,dx
         & = \int_0^1 \frac{x^{n+1}}{x + 5} \,dx + 5 \int_0^1 \frac{x^n}{x + 5} \,dx \\
        \frac{1}{n + 1} x^{n+1} \Big|_0^1
         & = y_{n+1} + 5 y_n                                                         \\
        y_{n+1}
         & = \frac{1}{n + 1} - 5 y_n
    \end{align*}

    Fortunately, \begin{align*}
        y_0
         & = \int_0^1 \frac{1}{x + 5} \,dx
         & = \ln (x + 5) \Big|_0^1         \\
         & = \ln 6 - \ln 5                 \\
         & = \ln \frac{6}{5}
        \doteq 0.182
    \end{align*}

    By the recurrence, \begin{align*}
        y_1 & = \frac{1}{1} - 5 y_0 \doteq 1 - 5 (0.182) = 0.0900      \\
        y_2 & = \frac{1}{2} - 5 y_1 \doteq 0.5 - 5 (0.0900) = 0.0500   \\
        y_3 & = \frac{1}{3} - 5 y_2 \doteq 0.333 - 5 (0.0500) = 0.0830 \\
        y_4 & = \frac{1}{4} - 5 y_3 \doteq 0.25 - 5 (0.0830) = -0.165
    \end{align*}

    Clearly, something went wrong. We have a negative value for \( y_4 \), which is impossible. We also have \( y_3 > y_2 \). The problem is that we are using floating point arithmetic, which is not exact. We are losing precision in our calculations.

    What if we leave \( y_0 \) as an unevaluated term?

    \begin{align*}
        y_1 & = 1 - 5y_0                  \\
        y_2 & = \frac{1}{2} - 5y_1        \\
            & = -\frac{9}{2} + 25y_0      \\
        y_3 & = \frac{1}{3} - 5y_2        \\
            & = \frac{137}{6} - 125y_0    \\
        y_4 & = \frac{1}{4} - 5y_3        \\
            & = -\frac{1367}{12} + 625y_0
    \end{align*}

    We approximated \( y_0 = \ln \frac{6}{5} \approx 0.182 \). We know that the true value of \( y_0 \in [0.1815, 0.1825] \). Another way to express \( y_0 \) is \( y_0 = 0.182 + E \), where \( | E | \leq 0.0005 = 5 \times 10^{-4} \) is the error in our approximation.

    Substituting this into the formula for \( y_4 \), we get \begin{align*}
        y_4 & = -\frac{1367}{12} + 625(0.182 + E) \\
            & = -113.91\dot{6} + 113.75 + 625E    \\
            & = -0.1\dot{6} + 625E
    \end{align*}
    where \[
        625 E \leq 625 \times 5 \times 10^{-4} = 0.3125
    \] and \[
        y_4 < y_0 \doteq 0.182
    \] so our propagated error is greater than the quantity to compute.
\end{example}

A lesson learned from the previous example is that the math textbook algorithms does not necessarily produce good computational algorithms. This algorithms for computing \( y_n \) is said to be an \term{numerically unstable algorithm}, since a small error was magnified by the algorithm. We want the algorithms to be \term{numerically stable}.

\begin{definition}[Numerically Unstable]\index{Numerically Unstable}
    An algorithm is said to be \term{numerically unstable} if the error in the output is not bounded by the error in the input.
\end{definition}

\begin{definition}[Numerical Stability]\index{Numerical Stability}
    An algorithm is said to be \term{numerically stable} if the error in the output is bounded by the error in the input.
\end{definition}