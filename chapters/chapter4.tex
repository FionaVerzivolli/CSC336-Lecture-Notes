\chapter{Linear systems}

\section{Introduction}

\begin{definition}[Non-Singular Matrix]\index{Non-Singular Matrix}
    A coefficient matrix \( A \) is \term{non-singular} if and only if \[
        \exists A^{-1} \text{ such that } A A^{-1} = A^{-1} A = I
    \]
\end{definition}

\begin{remark}
    Some facts
    \begin{itemize}
        \item \( A\vec{x} = \vec{b} \) has a \textbf{unique} solution if and only if \( A \) is non-singular.
        \item \( A \) is non-singular if and only if \( \det(A) \neq 0 \).
    \end{itemize}
\end{remark}

In the contrast, a singular matrix is one that is not invertible.

\begin{proposition}
    If \( \exists z \neq 0 \) such that \( A z = 0 \), then \( A \) is singular.
\end{proposition}

\begin{remark}
    Suppose exists \( y \) such that \( A \vec{y} = \vec{b} \), then \( y + \alpha z \) solves \( A \vec{x} = \vec{b} \) may not be a solution.

    For example, \[
        \begin{pmatrix} 1 & 2 \\ 3 & 4 \end{pmatrix} \begin{pmatrix} x_1 \\ x_2 \end{pmatrix} = \begin{pmatrix} 3 \\ 7 \end{pmatrix}
    \] has no solution.
\end{remark}

We will stick to the case where \( A \) is non-singular.

\subsection{Cramer's Rule}

\begin{definition}[Cramer's Rule]\index{Cramer's Rule}
    For a linear system \( A \vec{x} = \vec{b} \), the solution is \[
        x_i = \frac{\det(A_i)}{\det(A)}
    \] where \( A_i \) is the matrix obtained by replacing the \( i \)th column of \( A \) with \( b \).
\end{definition}

\begin{remark}
    Cramer's Rule is not practical for large systems, as it requires \( n+1 \) determinants.
\end{remark}

\section{Gaussian Elimination}

Alternatively, we can compute \( A^{-1} \) and then \( x = A^{-1} b \).

To solve for \( A^{-1} \), we solve for \( AY = I \), then \( x = Yb \).

For \( i = 1 \) to \( n \):
use some method to solve \( Ay_i = e_i = \begin{pmatrix} 0 \\ \vdots \\ 1 \\ \vdots \\ 0 \end{pmatrix} \).

\subsection{Intuition}

\subsubsection{Case I}

Suppose \( A \) is non-singular diagonal matrix, \[
    A = \begin{pmatrix}
        a_1    & 0      & \cdots & 0      \\
        0      & a_2    & \cdots & 0      \\
        \vdots & \vdots & \ddots & \vdots \\
        0      & 0      & \cdots & a_n
    \end{pmatrix}
\]

We must have \( \begin{cases}
    a_{ij} \neq 0 & \text{if } i = j    \\
    a_{ij} = 0    & \text{if } i \neq j
\end{cases} \)

Indeed, since \( A \) non-singular, \( \det(A) \neq 0 \), and since \( A \) diagonal, \( \det(A) = a_1 a_2 \cdots a_n \neq 0 \). None of the diagonal elements can be zero.

Then, for \( Ax = b \), we have \( x_i = b_i / a_{i,i} \). We can solve this within a single loop.

\begin{algorithmic}
    \For{\( i = 1 \) to \( n \)}
    \State \( x_i = b_i / a_{i,i} \)
    \EndFor
\end{algorithmic}

To count the number of operations, we define a new unit of operation, the \term{flop}.

\begin{definition}[FLOP]\index{FLOP}
    A \term{FLOP} is a \textbf{F}loating-point \textbf{L}inear \textbf{O}peration. This is a single arithmetic operation on floating-point numbers.
\end{definition}

\begin{example}
    For the above algorithm, each iteration require 1 FLOP -- a division. Thus, we require \( n \) FLOPs to solve the system.
\end{example}

\subsubsection{Case II}

Suppose \( A \) is non-singular lower-triangular, \[
    A = \begin{pmatrix}
        a_{11} & 0      & \cdots & 0      \\
        a_{21} & a_{22} & \cdots & 0      \\
        \vdots & \vdots & \ddots & \vdots \\
        a_{n1} & a_{n2} & \cdots & a_{nn}
    \end{pmatrix}
\]
We must have \( \begin{cases}
    a_{ij} = 0    & \text{if } i < j                     \\
    a_{ii} \neq 0 & \text{since } A \text{ non-singular}
\end{cases} \)

\begin{itemize}
    \item \( x_1 = b_1 / a_{1,1} \)
    \item \( x_2 = (b_2 - a_{2,1} x_1) / a_{2,2} \)
    \item \( x_3 = (b_3 - a_{3,1} x_1 - a_{3,2} x_2) / a_{3,3} \)
    \item \( \vdots \)
    \item \( \displaystyle x_i = \left( b_i - \sum_{j=1}^{i-1} a_{ij} x_j \right) / a_{ii} \)
\end{itemize}

This algorithm is called \term{forward substitution}.

\begin{algorithm}[H]
    \begin{algorithmic}
        \Function{ForwardSubstitution}{$A, b$}
        \For{\( i = 1 \) \To \( n \)}
        \State \( sum = 0 \)
        \For{\( j = 1 \) \To \( i-1 \)}
        \State \( sum = sum + a_{ij} \cdot x_j \)
        \EndFor
        \State \( x_i = (b_i - sum) / a_{ii} \)
        \EndFor
        \State \Return \( x \)
        \EndFunction
    \end{algorithmic}
\end{algorithm}

We have the operation count \[
    \sum_{i=1}^n \left[ \left( \sum_{j=1}^{i-1} 2 \right) + 2 \right] = n^2 + \Theta(n) \text{ FLOPs}.
\]

\subsubsection{Case III}

Suppose \( A \) is non-singular upper-triangular, \[
    A = \begin{pmatrix}
        a_{11} & a_{12} & \cdots & a_{1n} \\
        0      & a_{22} & \cdots & a_{2n} \\
        \vdots & \vdots & \ddots & \vdots \\
        0      & 0      & \cdots & a_{nn}
    \end{pmatrix}
\]

We must have \( \begin{cases}
    a_{ij} = 0    & \text{if } i > j                     \\
    a_{ii} \neq 0 & \text{since } A \text{ non-singular}
\end{cases} \)

\begin{itemize}
    \item \( x_n = b_n / a_{n,n} \)
    \item \( x_{n-1} = (b_{n-1} - a_{n-1,n} x_n) / a_{n-1,n-1} \)
    \item \( x_{n-2} = (b_{n-2} - a_{n-2,n} x_n - a_{n-2,n-1} x_{n-1}) / a_{n-2,n-2} \)
    \item \( \vdots \)
    \item \( \displaystyle x_i = \left( b_i - \sum_{j=i+1}^{n} a_{ij} x_j \right) / a_{ii} \)
\end{itemize}

This algorithm is called \term{backward substitution}.

\begin{algorithm}[H]
    \begin{algorithmic}
        \Function{BackwardSubstitution}{$A, b$}
        \For{\( i = n \) \DownTo \( 1 \)}
        \State \( sum = 0 \)
        \For{\( j = i+1 \) \To \( n \)}
        \State \( sum = sum + a_{ij} \cdot x_j \)
        \EndFor
        \State \( x_i = (b_i - sum) / a_{ii} \)
        \EndFor
        \State \Return \( x \)
        \EndFunction
    \end{algorithmic}
\end{algorithm}

The operation count will be the same as for forward substitution, \[
    n^2 + \Theta(n) \text{ FLOPs}.
\]

\subsubsection{Case IV}

What if \( A \) is dense? We can use Gaussian elimination to reduce \( A \) to a triangular form.

\subsection{Gaussian Elimination}

\begin{example}
    Suppose we want to solve the linear system \[
        \begin{pmatrix}
            2 & 1 & 1 & 0 \\
            4 & 3 & 3 & 1 \\
            8 & 7 & 9 & 5 \\
            6 & 7 & 9 & 8
        \end{pmatrix}
        \begin{pmatrix}
            x_1 \\ x_2 \\ x_3 \\ x_4
        \end{pmatrix}
        =
        \begin{pmatrix}
            1 \\ 1 \\ -1 \\ 3
        \end{pmatrix}
    \]

    Recall from linear algebra that elementary operations do not change the solution to the system. We can use the elementary row operation \[
        R_j = R_j - m \cdot R_i
    \] to convert \( A \) to an upper-triangular form.

    \begin{enumerate}
        \item \( R_2 = R_2 - 2 \cdot R_1 \qquad R_3 = R_3 - 4 \cdot R_1 \qquad R_4 = R_4 - 3 \cdot R_1 \)
              \[
                  \begin{pmatrix}
                      2 & 1 & 1 & 0 \\
                      0 & 1 & 1 & 1 \\
                      0 & 3 & 5 & 5 \\
                      0 & 4 & 6 & 8 \\
                  \end{pmatrix}
                  \vec{x}
                  =
                  \begin{pmatrix}
                      1 \\ -1 \\ -5 \\ -6
                  \end{pmatrix}
              \]

        \item \( R_3 = R_3 - 3 \cdot R_2 \qquad R_4 = R_4 - 4 \cdot R_2 \)

              \[
                  \begin{pmatrix}
                      2 & 1 & 1 & 0 \\
                      0 & 1 & 1 & 1 \\
                      0 & 0 & 2 & 2 \\
                      0 & 0 & 2 & 4 \\
                  \end{pmatrix}
                  \vec{x}
                  =
                  \begin{pmatrix}
                      1 \\ -1 \\ -2 \\ -2
                  \end{pmatrix}
              \]

        \item \( R_4 = R_4 - R_3 \)

              \[
                  \begin{pmatrix}
                      2 & 1 & 1 & 0 \\
                      0 & 1 & 1 & 1 \\
                      0 & 0 & 2 & 2 \\
                      0 & 0 & 0 & 2 \\
                  \end{pmatrix}
                  \vec{x}
                  =
                  \begin{pmatrix}
                      1 \\ -1 \\ -2 \\ 0
                  \end{pmatrix}
              \]
    \end{enumerate}

    We now have a triangular system. We can use backward substitution to solve for \( x \). \[
        \vec{x} = \begin{pmatrix}
            1 \\ 0 \\ -1 \\ 0
        \end{pmatrix}
    \]

    How could we program this algorithm?
\end{example}

\begin{algorithm}[H]
    \begin{algorithmic}[1]
        \Function{GaussianElimination}{$A, b$}
        \State \( n \gets \text{size}(A) \)
        \State \( ms \gets \texttt{[]} \) \Comment{Multipliers}
        \For{\( i \gets 1 \) \To \( n-1 \)} \Comment{Going down the diagonal}
        \For{\( j \gets i+1 \) \To \( n \)} \Comment{Below the diagonal}
        \State \( m \gets A_{j,i} / A_{i,i} \)
        \State \( ms[j,i] \gets m \)
        % \State \( A_{j,i} = 0 \)
        \For{\( k \gets i+1 \) \To \( n \)}
        \State \( A_{j,k} \gets A_{j,k} - m \cdot A_{i,k} \)
        \EndFor
        % \State \( b_j = b_j - m \cdot b_i \)
        \EndFor
        \EndFor
        \State \Return \( A, ms \)
        \EndFunction
    \end{algorithmic}
\end{algorithm}

\begin{remark}
    The reason not to update \( b \), but rather to store the multipliers, is that often in practice we have multiple right-hand sides. We can then use the same multipliers to solve for all right-hand sides.
\end{remark}

What is the operation count for Gaussian elimination? We consider only the operations on \( A \), since the operations on \( b \) are negligible.

\begin{itemize}
    \item The inner most operation occurs on line 9, with 2 FLOPs.
    \item They are repeated for \( k = i+1 \) to \( n \), \( \sum_{k=i+1}^n 2 \) FLOPs.
    \item There is one more operation on line 6, with 1 FLOP.
    \item The \( j \) loop is repeated for \( j = i+1 \) to \( n \), \( \sum_{j=i+1}^n \left( 1 + \sum_{k=i+1}^n 2 \right) \) FLOPs.
    \item The outer loop is repeated for \( i = 1 \) to \( n-1 \), \( \sum_{i=1}^{n-1} \left( 1 + \sum_{j=i+1}^n \left( 1 + \sum_{k=i+1}^n 2 \right) \right) \) FLOPs.
\end{itemize}

We then simplify
\begin{align*}
    \operatorname{FLOP}
     &
    = \sum_{i=1}^{n-1} \left( 1 + \sum_{j=i+1}^n \left( 1 + \sum_{k=i+1}^n 2 \right) \right)
    \\
     &
    = \sum_{i=1}^{n-1} \left( 1 + \sum_{j=i+1}^n \left( 1 + 2(n - i) \right) \right)
    \\
     &
    = \sum_{i=1}^{n-1} \left[ (n - i) (1 + 2(n - i)) \right]
\end{align*}

Taking \( m = n - i \), we have
\begin{align*}
    \operatorname{FLOP}
     &
    = \sum_{m=1}^{n-1} m (1 + 2m)
    \\
     &
    = \left( \sum_{m=1}^{n-1} m \right) + 2 \left( \sum_{m=1}^{n-1} m^2 \right)
    \\
     &
    = \frac{n(n-1)}{2} + 2 \cdot \frac{2n^3 - 3n^2 + n}{6}
    \\
     &
    = \frac{2n^3}{3} - \frac{n^2}{2} - \frac{n}{6}
    \\
     & = \frac{2}{3} n^3 + \Theta(n^2) \text{ FLOPs}
\end{align*}

\subsection{Gauss Transforms}

\begin{definition}[Gauss Transformation Matrix]
    A \term{\( n \times n \) Gauss transformation matrix} \( m_R \) is defined as \[
        m_R = I - \vec{m}_r \cdot \vec{e}_r^\top
    \] where \( m_R \in \R^n \) and \[
        \vec{m}_r = \begin{pmatrix}
            0 \\ \vdots \\ 0 \\ m_{r+1,r} \\ m_{r+2,r} \\ \vdots \\ m_{n,r}
        \end{pmatrix}
        \qquad \text{and} \qquad
        \vec{e}_r = \begin{pmatrix}[l]
            0 \\ \vdots \\ 0 \\ 1 \text{ \(r\)th row} \\ 0 \\ \vdots \\ 0
        \end{pmatrix}
    \]
\end{definition}

\begin{remark}
    \begin{align*}
        M_r
         &
        = I - m_r e_r^\top
        \\
         &
        = I - \begin{pmatrix}
                  0 &        &           &   &   &        &   \\
                    & \ddots &           &   &   &        &   \\
                    &        & 0         &   &   &        &   \\
                    &        & m_{r+1,r} & 0 &   &        &   \\
                    &        & m_{r+2,r} & 0 & 0 &        &   \\
                    &        & \vdots    &   &   & \ddots &   \\
                    &        & m_{n,r}   &   &   &        & 0
              \end{pmatrix}
        % \\
        %  &
        = \begin{pmatrix}
              1 &        &            &   &   &        &   \\
                & \ddots &            &   &   &        &   \\
                &        & 1          &   &   &        &   \\
                &        & -m_{r+1,r} & 1 &   &        &   \\
                &        & -m_{r+2,r} & 0 & 1 &        &   \\
                &        & \vdots     &   &   & \ddots &   \\
                &        & -m_{n,r}   &   &   &        & 1
          \end{pmatrix}
    \end{align*}
\end{remark}

\begin{remark}
    Gauss transforms have nice properties:
    \begin{enumerate}
        \item For indices \( r \leq s \), \[
                  M_r M_s = I - \vec{m}_r \vec{e}_r^\top - \vec{m}_s \vec{e}_s^\top
              \] (not a Gauss transform)

        \item \[
                  M_r^{-1} = I + \vec{m}_r \vec{e}_r^\top
              \] (is a Gauss transform)

        \item For any \(n-\)vector \( \vec{v} = [v_1, v_2, \ldots, v_n]^\top \), \begin{align*}
                  M_r \vec{v}
                   &
                  = ( I - \vec{m}_r \vec{e}_r^\top) \vec{v}
                  \\
                   &
                  = \vec{v} - \vec{m}_r (\vec{e}_r^\top \vec{v})
                  \\
                   &
                  = \vec{v} - v_r \vec{m}_r
                  \\
                   &
                  = \begin{pmatrix}
                        v_1 \\ \vdots \\ v_{r} \\ v_{r+1} - v_r m_{r+1,r} \\ \vdots \\ v_n - v_r m_{n,r}
                    \end{pmatrix}
              \end{align*}
    \end{enumerate}
\end{remark}

\begin{example}
    Suppose \[
        A = \begin{pmatrix}
            2 & 1 & 1 & 0 \\ 4 & 3 & 3 & 1 \\ 8 & 7 & 9 & 5 \\ 6 & 7 & 9 & 8
        \end{pmatrix}
    \]

    \begin{itemize}
        \item Choose \[
                  m_1 = \begin{pmatrix}
                      0 \\ 2 \\ 4 \\ 3
                  \end{pmatrix}
                  \implies
                  M_1 = \begin{pmatrix}
                      1 & 0 & 0 & 0 \\ -2 & 1 & 0 & 0 \\ -4 & 0 & 1 & 0 \\ -3 & 0 & 0 & 1
                  \end{pmatrix}
              \]

              We have \[
                  M_1 A
                  =
                  \begin{pmatrix}
                      1 & 0 & 0 & 0 \\ -2 & 1 & 0 & 0 \\ -4 & 0 & 1 & 0 \\ -3 & 0 & 0 & 1
                  \end{pmatrix}
                  \begin{pmatrix}
                      2 & 1 & 1 & 0 \\ 4 & 3 & 3 & 1 \\ 8 & 7 & 9 & 5 \\ 6 & 7 & 9 & 8
                  \end{pmatrix}
                  =
                  \begin{pmatrix}
                      2 & 1 & 1 & 0 \\ 0 & 1 & 1 & 1 \\ 0 & 3 & 5 & 5 \\ 0 & 4 & 6 & 8
                  \end{pmatrix}
              \]

        \item Choose \[
                  m_2 = \begin{pmatrix}
                      0 \\ 0 \\ 3 \\ 4
                  \end{pmatrix}
                  \implies
                  M_2 = \begin{pmatrix}
                      1 & 0 & 0 & 0 \\ 0 & 1 & 0 & 0 \\ 0 & -3 & 1 & 0 \\ 0 & -4 & 0 & 1
                  \end{pmatrix}
              \]

              We have \[
                  M_2 M_1 A
                  =
                  \begin{pmatrix}
                      1 & 0 & 0 & 0 \\ 0 & 1 & 0 & 0 \\ 0 & -3 & 1 & 0 \\ 0 & -4 & 0 & 1
                  \end{pmatrix}
                  \begin{pmatrix}
                      2 & 1 & 1 & 0 \\ 0 & 1 & 1 & 1 \\ 0 & 3 & 5 & 5 \\ 0 & 4 & 6 & 8
                  \end{pmatrix}
                  =
                  \begin{pmatrix}
                      2 & 1 & 1 & 0 \\ 0 & 1 & 1 & 1 \\ 0 & 0 & 2 & 2 \\ 0 & 0 & 2 & 4
                  \end{pmatrix}
              \]

        \item Choose \[
                  m_3 = \begin{pmatrix}
                      0 \\ 0 \\ 0 \\ 1
                  \end{pmatrix}
                  \implies
                  M_3 = \begin{pmatrix}
                      1 & 0 & 0 & 0 \\ 0 & 1 & 0 & 0 \\ 0 & 0 & 1 & 0 \\ 0 & 0 & -1 & 1
                  \end{pmatrix}
              \]

              We have \[
                  M_3 M_2 M_1 A
                  =
                  \begin{pmatrix}
                      1 & 0 & 0 & 0 \\ 0 & 1 & 0 & 0 \\ 0 & 0 & 1 & 0 \\ 0 & 0 & -1 & 1
                  \end{pmatrix}
                  \begin{pmatrix}
                      2 & 1 & 1 & 0 \\ 0 & 1 & 1 & 1 \\ 0 & 0 & 2 & 2 \\ 0 & 0 & 2 & 4
                  \end{pmatrix}
                  =
                  \begin{pmatrix}
                      2 & 1 & 1 & 0 \\ 0 & 1 & 1 & 1 \\ 0 & 0 & 2 & 2 \\ 0 & 0 & 0 & 2
                  \end{pmatrix}
              \]
    \end{itemize}

    We end up with an upper-triangular matrix \[
        U = M_3 M_3 M_1 A = \begin{pmatrix}
            2 & 1 & 1 & 0 \\ 0 & 1 & 1 & 1 \\ 0 & 0 & 2 & 2 \\ 0 & 0 & 0 & 2
        \end{pmatrix}
    \]

    This also implies \begin{align*}
        A = (M_3 M_2 M_1)^{-1} U
         &
        = M_1^{-1} M_2^{-1} M_3^{-1} U
        \\
         &
        = (I + \vec{m}_1 \vec{e}_1^\top) (I + \vec{m}_2 \vec{e}_2^\top) (I + \vec{m}_3 \vec{e}_3^\top) U
         & \text{by property 2}
        \\
         &
        =
        \begin{pmatrix}
            1 & 0 & 0 & 0 \\ 2 & 1 & 0 & 0 \\ 4 & 3 & 1 & 0 \\ 3 & 4 & 1 & 1
        \end{pmatrix} U
    \end{align*}
    and we have an anti lower-triangular matrix \[
        L = \begin{pmatrix}
            1 & 0 & 0 & 0 \\ 2 & 1 & 0 & 0 \\ 4 & 0 & 1 & 0 \\ 3 & 0 & 0 & 1
        \end{pmatrix}
    \]

    We can think of Gaussian Elimination as factoring \( A \) into \( LU \), where \( L \) is lower-triangular and \( U \) is upper-triangular.

    Verify: \[
        LU =
        \begin{pmatrix}
            1 & 0 & 0 & 0 \\ 2 & 1 & 0 & 0 \\ 4 & 3 & 1 & 0 \\ 3 & 4 & 1 & 1
        \end{pmatrix}
        \begin{pmatrix}
            2 & 1 & 1 & 0 \\ 0 & 1 & 1 & 1 \\ 0 & 0 & 2 & 2 \\ 0 & 0 & 0 & 2
        \end{pmatrix}
        =
        \begin{pmatrix}
            2 & 1 & 1 & 0 \\ 4 & 3 & 3 & 1 \\ 8 & 7 & 9 & 5 \\ 6 & 7 & 9 & 8
        \end{pmatrix}
    \]
\end{example}

\section{Solving System of Linear Equations}

\subsection{LU Factorization}

How to solve \( A \vec{x} = \vec{b} \)?

\begin{enumerate}
    \item Factor \( A = LU \)

          The operation cont is \[
              \frac{2}{3} n^3 + \Theta(n^2) \text{ FLOPs}
          \]

          Then, we have \[
              L(U \vec{x}) = \vec{b}
          \]

          Let \( \vec{y} = U\vec{x} \), we have an triangular system \[
              L \vec{y} = \vec{b}
          \]

    \item Solve \( L \vec{y} = \vec{b} \) for \( \vec{y} \) using forward substitution.

          The operation count is \[
              n^2 + \Theta(n) \text{ FLOPs}
          \]

          Then, we have \[
              U \vec{x} = \vec{y}
          \]

    \item Solve \( U \vec{x} = \vec{y} \) for \( \vec{x} \) using backward substitution.

          The operation count is \[
              n^2 + \Theta(n) \text{ FLOPs}
          \]

          The total operation count is \[
              n^2 + \Theta(n) \text{ FLOPs}
          \]

    \item The total cost is \[
              \frac{2}{3} n^3 + \Theta(n^2) \text{ FLOPs}
          \]
\end{enumerate}

\begin{remark}
    Now, suppose we are given a new problem \[
        A \vec{z} = \vec{c} \qquad \text{for } \vec{z}
    \] where \( A \) is the same as before, but \( \vec{c} \) is different.

        {~~~}

    Note that the output from step 1 depends solely on \( A \), and we do not need to recompute it. We \textit{skip the most expensive step}, and use the same \( L \) and \( U \) to solve for \( \vec{z} \) using forward and backward substitution.

    \begin{itemize}
        \item Solve \( L \vec{d} = \vec{c} \) for \( \vec{d} \) using forward substitution
        \item Solve \( U \vec{z} = \vec{d} \) for \( \vec{z} \) using backward substitution
    \end{itemize}

    {~~~}

    The total new cost is \[
        2n^2 + \Theta(n) \text{ FLOPs}
    \]
\end{remark}

\begin{remark}
    How to reduce memory usage?

    Note that
    \begin{itemize}
        \item \( L \) is unit lower triangular (diagonal is all 1s)
        \item \( U \) is upper triangular, and
        \item \( A \) is dense.
    \end{itemize}

    We can store \( L \) and \( U \) in a single matrix, for example, \[
        \begin{pmatrix}
            \color{blue} 2 & \color{blue} 1 & \color{blue} 1 & \color{blue} 0 \\
            \color{red} 1  & \color{blue} 1 & \color{blue} 1 & \color{blue} 1 \\
            \color{red} 4  & \color{red} 3  & \color{blue} 2 & \color{blue} 2 \\
            \color{red} 3  & \color{red} 4  & \color{red} 1  & \color{blue} 2
        \end{pmatrix}
    \] where \[
        {\color{red} L = \begin{pmatrix}
                    1 & 0 & 0 & 0 \\
                    2 & 1 & 0 & 0 \\
                    4 & 3 & 1 & 0 \\
                    3 & 4 & 1 & 1
                \end{pmatrix}}
        \qquad \text{ and } \qquad
        {\color{blue} U = \begin{pmatrix}
                2 & 1 & 1 & 0 \\
                0 & 1 & 1 & 1 \\
                0 & 0 & 2 & 2 \\
                0 & 0 & 0 & 2
            \end{pmatrix}}
    \]
\end{remark}

\subsubsection{Edge-case}

\begin{example}
    Suppose we want to compute the LU factorization of \[
        \begin{pmatrix}
            0 & 1 \\ 1 & 0
        \end{pmatrix}
    \]

    The algorithm fails from 0 division \[
        m_{1,1} = \frac{1}{0}
    \]

    However, we could perform a row exchange \( R_1 \leftrightarrow R_2 \) to get \[
        \begin{pmatrix}
            1 & 0 \\ 0 & 1
        \end{pmatrix}
    \] which is now solvable.

    We can use permutation matrix to represent row exchanges. For example, \[
        P = \begin{pmatrix}
            0 & 1 \\ 1 & 0
        \end{pmatrix}
    \] represents the row exchange \( R_1 \leftrightarrow R_2 \).
\end{example}